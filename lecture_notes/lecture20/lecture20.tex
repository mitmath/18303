% !TEX TS-program = xelatex
% !TEX encoding = UTF-8 Unicode
\input{../structure.tex}

\title{Advection problems vol. 2}
\subtitle{Flow equations etc.}
\date{25/3/2021}
\date{}
%\author{18.303 Linear Partial Differential Equations: Analysis and Numerics}
\institute{18.303 Linear Partial Differential Equations: Analysis and Numerics}
\titlegraphic{\hfill\includegraphics[height=2em]{../MIT-logo.pdf}}

\begin{document}
	
	\maketitle
	
%	\begin{frame}{Table of contents}
%		\setbeamertemplate{section in toc}[sections numbered]
%		\tableofcontents%[hideallsubsections]
%	\end{frame}

\begin{frame}{Method of characteristics}
	Consider a differential equation
	\[ a(x,y,z) z_x + b(x,y,z) z_y = c(x,y,z), \]
	where $ a $, $ b $, and $ c $ are known functions.
	We want to solve for $ z(x,y) $. 
	
	\pause
	The solution can be comprised of \alert{characteristic curves} that are parametrized curves in $ \R^3 $ satisfying 
	\[  
	\begin{split}
		\dot{x}(t) &= a(x,y,z), \\
		\dot{y}(t) &= b(x,y,z), \\
		\dot{z}(t) &= c(x,y,z).
	\end{split}
	\]
	
	\pause
	Substituting the equations in the original PDE gives
	\[ \dot{x}(t) z_x + \dot{y}(t) z_y = \dot{z}(t), \]
	which is just the chain rule for taking the derivative $ \dot{z} $. We see that this is consistent as long as we have some known initial point $ z(x(t_0),y(t_0)) $.
\end{frame}

\begin{frame}
	Another way of looking at this is to calculate the normal to the surface $ z(x,y) $. We get a normal by calculating the cross product of two tangent vectors for the points $ (x,y,z) $ given by 
	\[ \partial_x (x,y,z) = (1,0,z_x), \; \partial_y (x,y,z) = (0,1,z_y). \]
	
	\pause
	The normal is given by
	\[ (0,1,z_y) \times (1,0,z_x) = \vec{e}_y \times \vec{e}_x + z_x \vec{e}_y \times \vec{e}_z + z_y \vec{e}_z \times \vec{e}_x = (z_x,z_y,-1).  \]
	
	\pause
	Calculating the dot product 
	\[ (a,b,c) \cdot (z_x,z_y,-1) = a z_x + b z_y - c = 0 \]
	because of the original PDE. This shows that the vector field $ (a,b,c) $ is tangent to the surface $ z(x,y) $. Therefore we can parametrize a curve on $ z $ such that its tangent is $ \partial_t(x(t),y(t),z(t)) = (a,b,c) $.
\end{frame}

\begin{frame}{Burgers' equation}
	We have
	\[ \partial_t u(t,x) + u(t,x)\partial_x u(t,x) = 0 \]
	with some initial condition $ u(0,x) = u_{0}(x) $. Here $ t \geq 0 $ and $ x \in \R $.
	
	\pause
	Using the method of characteristics we find the characteristic curves 
	\[  
	\begin{split}
		\partial_\xi t(\xi) & = 1, \\
		\partial_\xi x(\xi) & = u(t(\xi),x(\xi)), \\
		\partial_\xi u(t(\xi),x(\xi)) &= 0. 
	\end{split}
	\]
	
	\pause
	We can solve the first equation by integrating $ \xi $ from 0 to $ \xi $ giving $ t(\xi) = \xi + t(0) $. We see that we can choose the parameter $ \xi = t $ since $ \xi $ and $ t $ are linearly related.   
	
\end{frame}

\begin{frame}
	The third equation gives us $ \partial_t u = 0 $. This means that the function $ u(t,x(t)) $ is a constant, i.e. the velocity doesn't vary along the trajectory $ (t,x(t)) $. Furthermore, from the initial condition we know that $ u(t,x(t)) = u(0,x(0)) = u_0(x(0)) = u_0(x_0) $.
	
	\pause
	The second equation gives 
	\[ \partial_t x(t) = u(t,x(t)). \]
	
	\pause
	This just means that $ u $ is indeed the velocity at $ (t,x(t)) $. Substituting the solution for $ u(t,x(t)) $ gives
	\[ \partial_t x(t) = u_0(x_0), \]
	which can be integrated from $ 0 $ to $ t $ giving 
	\[ x(t) = t u_0(x_0) + x_0. \]
\end{frame}

\begin{frame}
	We learn that the characteristic curves are actually two-dimensional lines $ (t,t u_0(x_0) + x_0) $ along which the velocity field is a constant.
	
	\pause
	These curves are parametrized by the initial point $ x_0 $. So, what is $ u(t,x) $ given any $ (t,x) $. We have the map 
	\[ x(t;x_0) = t u_0(x_0) + x_0, \]
	which can be formally inverted to give $ x_0(t,x(t)) $. Now the solution is given by $ u(t,x) = u_0(x_0(t,x(t))) $.
	
	\pause
	Can $ x(t;x_0) $ really be inverted? It's possible if $ x(t;x_0) $ is a \emph{unique} function of $ x_0 $ i.e. for any pair $ x, y $ we have $ x(t;x_0) \neq x(t;x_1) $, when $ x_0 \neq x_1 $.
	
	\pause
	The function $ x(t;x_0) $ is a continuous map (as long as $ u_0 $ is continuous) so in case there are no crossings it preserves the order i.e. if $ x_0 < x_1 <x_2 $ it implies that $ x(t;x_0) < x(t;x_1) < x(t;x_2) $ for any $ t $. 
	
	\pause
	Therefore it suffices to see if $ x(t;x_0) < x(t;x_0 + \epsilon)$ for some $ \epsilon >0 $.   
\end{frame}

\begin{frame}
	Assume there are no crossings i.e. $ x(t;x_0) < x(t;x_0 + \epsilon)$. Substituting the solution for $ x $ gives
	\[ t u_0(x_0) + x_0 < t u_0(x_0 + \epsilon) + x_0 + \epsilon. \]
	
	\pause
	Simplifying a bit gives
	\[ -t \frac{u_0(x_0 + \epsilon) - u_0(x_0)}{\epsilon} < 1. \]
	Taking $ \epsilon \to 0 $ gives a derivative and we have
	\[ u_0'(x_0) > -\frac{1}{t}. \]
	
	\pause
	We see that there \emph{will} be a crossing with the crossing time $ t = -1/u_0'(x_0) $ if $ u_0'(x_0)<0 $. 
\end{frame}

\begin{frame}
	What does this mean in practice?
	
	\pause
	To answer this question let's calculate
	\[ \partial_x u(t,x) = \pfrac{x_0}{x} u_0'(x_0). \]
	
	\pause
	From the equation for $ x $ we get 
	\[ \pfrac{x}{x_0} = t u_0'(x_0) + 1 = \left( \pfrac{x_0}{x} \right)^{-1}. \] 
	
	\pause
	Substituting gives 
	\[ \partial_x u(t,x) = \frac{u_0'(x_0)}{t u_0'(x_0) + 1}. \]
	
	\pause
	This approaches infinity as $ t u_0'(x_0) \to -1 $ i.e. crossing of the characteristic curves just means that the spatial derivatives of the velocity field blow up. This is known as \alert{shock formation}.
\end{frame}

\begin{frame}
	Another way to see this is to define a density. Assume we have a uniform distribution of something along the $ x $ coordinate. How much stuff do we have between $ 0 $ and $ x $ at time $ t $?
	
	\pause
	The answer is the length of the line segment $ [x_1,x_2] $ for which
	\[ 0 = t u_0(x_1) + x_1, \; x = t u_0(x_2) + x_2 \]
	i.e. the stuff that is transported between points $ 0 $ and $ x $ at time $ t $ by the flow. Mathematically we have 
	\[ \text{stuff} = \int_{x_1}^{x_2} \diff x_0 = \int_{0}^{x} \pfrac{x_0}{x} \diff x. \]
	
	\pause
	Therefore 
	$$ \pfrac{x_0}{x} = \frac{1}{t u_0'(x_0) + 1} $$ 
	can be seen as a \emph{density} that approaches infinity when a blow-up occurs. This means that stuff gets packed to a single point.
\end{frame}

\begin{frame}{18.303 summary}
	The most important takeaway from this class is that functions can be seen as \emph{vectors} and differentials as \emph{linear operators}. Therefore we can always express our functions in a suitable basis.
	
	\pause
	These transformations between bases can be used analytically and numerically. For example if
	\[ \partial_t f(t,\vx) = \opone f(t,\vx). \]
	with some boundary and initial conditions, we can express $ f $ in the eigenbasis of $ \opone $ given by the differential equation
	\[ \opone \psi(\vx) = \lambda \psi(\vx) \]
	with the same boundary conditions for $ \vx $.
	
	\pause
	After such a change of basis it is almost trivial to solve for the time evolution. The eigenbasis can be usually found analytically for symmetric domains and simple enough operators.
\end{frame}

\begin{frame}{Transforms}
	We introduced a bunch of different transformations that can be seen as a change of bases.
	
	\pause
	Among these we have 
	\begin{itemize}[<+->]
		\item Fourier transform for functions on the real line (or $ \R^N $)
		\item Laplace transform for functions on $ \R_+ $
		\item Fourier series for periodic data on a bounded interval 
		\item Sine series for Dirichlet problems on bounded intervals
		\item Discrete Fourier transform for discrete data
	\end{itemize}
\end{frame}

\begin{frame}{Different numerical schemes}
	\begin{itemize}[<+->]
		\item \textbf{Finite difference methods:} easy to implement and very versatile.
		\item \textbf{Spectral methods (especially in Fourier basis):} will evaluate derivatives very accurately but cannot be used for arbitrary domains and boundary conditions (at least easily).
		\item \textbf{Finite element methods:} a bit harder to implement but efficient for complicated domains and boundary conditions. 
		\item Many great commercial and open source software packages exist for all the different methods.
		\item In the end we just have a bunch of linear algebra problems that computers can solve efficiently using existing packages.
	\end{itemize}
\end{frame}

\begin{frame}{Some analytical techniques}
	\begin{itemize}[<+->]
		\item Separation of variables for partial differential equations.
		\item Green's functions (there's a lot more to these than what we covered in this class)
		\item Method of characteristics for non-linear first order PDEs
		\item We also covered basic properties of distributions and how they can be used to make things simple
	\end{itemize}
\end{frame}

\begin{frame}{Map between linear algebra and differential equations}
	\textbf{Homogeneous problems:}
	\[ \opone u(\vx) = 0 \to A\vu = 0. \]
	
	\pause
	\textbf{Inhomogeneous problems:}
	\[ \opone u(\vx) = f(\vx) \to A \vu = \fone. \]
	
	\pause
	\textbf{Eigenproblems}
	\[ \opone \eigtwo = \lambda_{\vn} \eigtwo \to A \bm{\phi}_{\vn} = \lambda_{\vn}\bm{\phi}_{\vn}. \]
\end{frame}

%\begin{itemize}[<+->]
%\end{itemize}

\end{document}
