% !TEX TS-program = xelatex
% !TEX encoding = UTF-8 Unicode
\input{../structure.tex}

\title{A bit of vector calculus}
\subtitle{Some integration rules}
\date{25/3/2021}
\date{}
%\author{18.303 Linear Partial Differential Equations: Analysis and Numerics}
\institute{18.303 Linear Partial Differential Equations: Analysis and Numerics}
\titlegraphic{\hfill\includegraphics[height=2em]{../MIT-logo.pdf}}

\begin{document}
	
	\maketitle
	
%	\begin{frame}{Table of contents}
%		\setbeamertemplate{section in toc}[sections numbered]
%		\tableofcontents%[hideallsubsections]
%	\end{frame}

\begin{frame}{Divergence and gradient theorems}
	For $ g: \R^N \to \R $ we have 
	\[ \int_{\gamma} \nabla g(\vx) \cdot \diff \vx = g(\vx_{\text{e}})-g(\vx_{\text{s}}). \]
	Here $ \gamma $ is a differentiable path embedded in $ \R^N $ and $ g(\vx_{\text{s}}) $ and $ g(\vx_{\text{e}}) $ are the start and endpoints of the path. This is called the \alert{gradient theorem} or the fundamental theorem of calculus for line integrals.
	
	\pause
	For a compact $ \Omega \subset \R^N $ and a function $ \fone : \R^N \to \R^N $ we have
	\[ \int_{\Omega} \nabla \cdot \fone(\vx) \diff V = \int_{\partial \Omega } \fone(\vx) \cdot \diff \mathbf{S} = \int_{\partial \Omega } \fone(\vx) \cdot \hat{\vn}(\vx) \diff S, \]
	where $ \hat{\vn} $ is the unit normal of a given point on the boundary of the region $ \Omega $. We assume here that the boundary is piecewise smooth. This is called the \alert{divergence theorem} or Gauss's theorem.
\end{frame}

\begin{frame}{Stokes' theorem}
	Let $ \vec{A}: \R^3 \to \R^3 $ be a sufficiently smooth vector field. Let $ \Sigma $ be a \emph{simply connected} smooth two-dimensional subset of $ \R^3 $ with a piecewise smooth boundary. 
	
	\pause
	We have
	\[ \int_{\Sigma} \nabla \times \vec{A} \cdot \diff \vec{S} = \oint_{\partial \Sigma} \vec{A} \cdot \diff \vx. \]
	Here the integral is contracted against the the tangent . 
\end{frame}

\begin{frame}{Green's first identity}
	Let $ \varphi, \psi: \R^N \to \R $ be differentiable functions and $ \Omega \subset \R^N $ a region with a piecewise differentiable boundary. 
	
	\pause
	\[ \int_{\Omega} \psi \Delta \varphi + \nabla \psi \cdot \nabla \varphi \diff V  
	= \int_{\partial \Omega} \psi \nabla \varphi \cdot \diff \vec{S}.
	\]
	
	\pause
	This can be also written as 
	\[  \int_{\Omega} \nabla \psi \cdot \nabla \varphi \diff V 
	= \int_{\partial \Omega} \psi \nabla \varphi \cdot \diff \vec{S} - 
	\int_{\Omega} \psi \Delta \varphi  \diff V,
	\]
	so it becomes a rule for integration by parts for vector calculus.
\end{frame}

\begin{frame}{Exercise}
	Derive Green's first identity using the divergence theorem. Hint, choose $ \fone = \psi \nabla \varphi $.
\end{frame}

\begin{frame}
	We can use the divergence theorem to derive many other useful identities. 
	
	\pause
	E.g. choosing $ \fone = \psi \vec{I} $, where $ \vec{I} $ is the identity matrix gives 
	\[ \int_{\Omega} \nabla \cdot (\psi \vec{I}) \diff V = \int_{\Omega} \nabla\psi \diff V = \int_{\partial \Omega} \psi \vec{I} \cdot \diff \vec{S} = \int_{\partial \Omega} \psi  \diff \vec{S}. \]
\end{frame}

%\begin{itemize}[<+->]
%\end{itemize}

\end{document}
