% !TEX TS-program = xelatex
% !TEX encoding = UTF-8 Unicode
\input{../structure.tex}

\title{Advection problems}
\subtitle{Flow equations etc.}
\date{25/3/2021}
\date{}
%\author{18.303 Linear Partial Differential Equations: Analysis and Numerics}
\institute{18.303 Linear Partial Differential Equations: Analysis and Numerics}
\titlegraphic{\hfill\includegraphics[height=2em]{../MIT-logo.pdf}}

\begin{document}
	
	\maketitle
	
%	\begin{frame}{Table of contents}
%		\setbeamertemplate{section in toc}[sections numbered]
%		\tableofcontents%[hideallsubsections]
%	\end{frame}

\begin{frame}{Elementary advection problem}
	Let $ u: \R_{+} \times \R \to \R $. This is a $ 1+1 $ dimensional field that is varying in time and space. We will look at a differential equation
	\[ u_t(t,x) + c u_x(t,x) = 0 \]
	with some initial condition $ u(0,x) = u_{0}(x) $. Here $ c \in \R $ is a model parameter.
	
	\pause
	Let us make a change of coordinates $ x = y + ct $. $ y $ will be the coordinate that is moving at a constant velocity $ c $. Let $ v(t,y) = u(t,x) $. 
	
	\pause
	Using the chain rule gives
	\[ v_t(t,y) + \dd{y}{t} v_y(t,y) + c \dd{y}{x} v_y(t,y) = v_t(t,y) - c v_y(t,y) + c v_y(t,y) = 0. \]
	
	\pause
	We see that the solution is given by $ v_t(t,y) = 0 $, which can be solved by integrating from 0 to $ t $ giving
	\[ v(t,y) = u_{0}(y). \]
	
	\pause
	This means that $ v $ is constant at every point $ y $. The solution is given by the packet $ u_{0}(x) $ moving with a velocity $ c $. This can be also written as $ u(t,x) = u_0(x - ct). $
\end{frame}

\begin{frame}{Discrete Fourier transform}
	In order to talk about the numerical stability of the advection problem we'll introduce the \alert{discrete Fourier transform}. Assume we have data $ x_j = x(\Delta x j) $ on some interval and a function $ f(x_j) = f_j $, where $ j= 0,1,...,N-1 $. We define the discrete Fourier transform as
	\[ \hat{f}_j = \sum_{n=0}^{N-1} e^{-2\pi i nj/N} f_{n}. \]
	
	\pause
	The inverse transform is given by
	\[ f_{j} = \frac{1}{N} \sum_{n=0}^{N-1} e^{2\pi i n j/N} \hat{f}_{n}. \]
	
	\pause
	We can show that this works by inserting the definition for $ \hat{f} $:
	\[ \frac{1}{N} \sum_{n=0}^{N-1} e^{2\pi i n j/N} \sum_{k=0}^{N-1} e^{-2\pi i nk/N} f_{k}. \]
\end{frame}

\begin{frame}
	Changes the order of the summation gives
	\[ \frac{1}{N} \sum_{k=0}^{N-1} f_{k} \sum_{n=0}^{N-1} e^{2\pi i n (j-k)/N}.  \]
	
	\pause
	We notice that $ j-k $ is an integer. The latter sum is a geometric series so we have 
	\[ \sum_{n=0}^{N-1} e^{2\pi i n (j-k)/N} = \frac{1 -e^{2\pi i N (j-k)/N} }{1 - e^{2\pi i (j-k)/N}}. \]
	
	\pause
	The enumerator gives $ 1 - \exp(2\pi i(j-k)) = 0 $ if $ j-k \neq 0 $. If $ j-k=0 $ this will be a sum of ones giving $ N $. Altogether we have
	\[ \sum_{n=0}^{N-1} e^{2\pi i n (j-k)/N} = N \delta_{j}^{k}. \]
	
	\pause
	Now 
	\[ \sum_{k=0}^{N-1} f_{k} \delta_{j}^{k} = f_{j} \]
	just as we want.
\end{frame}

\begin{frame}
	The discrete Fourier transform is one of the most useful numerical tools. There is an algorithm called the \alert{fast Fourier transform} (FFT) that is readily implemented in all major programming languages and can be used to calculate the discrete Fourier transform efficiently. This is especially useful for periodic data. 
	
	\pause
	Note: the FFT can be also used to transform the functions into cosine or sine bases for solving e.g. Dirichlet problems. 
\end{frame}

\begin{frame}{Back to the advection problem}
	The advection problem we introduced was solved almost trivially but it turns out that it is a very hard problem for numerical treatment.
	
	\pause
	Let's assume that we solve the problem for periodic boundaries. The analytical solution is basically the same: the initial data $ u_{0}(x) $ is moved forward (or backward) with a constant speed $ c $. Let's write an implicit finite difference scheme 
	\[ \frac{u_{j}^{(n+1)} - u_{j}^{(n)}}{s} = -c \frac{u_{j + 1}^{(n)} - u_{j}^{(n)}}{h}.   \]
	
	\pause
	Shifting things around a bit gives
	\[ u_{j}^{(n+1)} = (1 + \sigma ) u_{j}^{(n)} - \sigma u_{j + 1}^{(n)}, \]
	where $ \sigma = cs/h $.
	
	\pause
	Next we will take the discrete Fourier transform.
\end{frame}

\begin{frame}
	The discrete Fourier transform is a linear transformation but in order to continue we need the discrete Fourier transform of $ u_{j+1}^{(n)} $. We have
	\[ u_{j+1}^{(n)} = \frac{1}{N} \sum_{k=0}^{N-1} e^{2\pi i k(j+1)/N} \hat{u}_{k}^{(n)}  
	= \frac{1}{N} \sum_{k=0}^{N-1} e^{2\pi i kj/N} e^{2\pi i k/N} \hat{u}_{k}^{(n)}. \]
	
	\pause
	We see that the discrete Fourier transform is 
	\[  \hat{u}_{k+1}^{(n)} = e^{2\pi i k/N} \hat{u}_{k}^{(n)}.  \]
	
	\pause
	Similarly,
	\[  \hat{u}_{k-1}^{(n)} = e^{-2\pi i k/N} \hat{u}_{k}^{(n)}.  \]
\end{frame}

\begin{frame}
	Inserting this result gives
	\[ \hat{u}_{k}^{(n+1)} = (1 + \sigma ) \hat{u}_{k}^{(n)} - \sigma e^{2\pi i k/N} \hat{u}_{k}^{(n)} 
	= (1 + \sigma - \sigma e^{2\pi i k/N} )\hat{u}_{k}^{(n)}.
	\]
	
	\pause
	This algorithm is unstable if $ |1 + \sigma - \sigma e^{2\pi i k/N} | > 1 $. Let's calculate
	\[ |1 + \sigma - \sigma e^{2\pi i k/N} |^2 = (1+\sigma)^2 - (1+\sigma)\sigma( e^{2\pi i k/N} + e^{-2\pi i k/N} ) + \sigma^2 = 1 + 2\sigma + 2\sigma^2 - 2(1+\sigma)\sigma \cos(2\pi i k /N). \]
	
	\pause
	In the worst possible case $ \cos(2\pi i k /N) = -1 $ (when $ k \approx N/2 $). We get 
	\[ 1 + 2\sigma + 2\sigma^2 + 2(1+\sigma)\sigma \leq 1,  \]
	which is solved by
	\[ -1 \leq \sigma \leq 0 \Leftrightarrow -1 \leq \frac{sc}{h} \leq 0. \]
	
	\pause
	We notice that the system is unconditionally unstable if $ c>0 $!
\end{frame}

\begin{frame}
	We can repeat the calculation for an explicit scheme
	\[ \frac{u_{j}^{(n+1)} - u_{j}^{(n)}}{s} = -c \frac{u_{j}^{(n)} - u_{j-1}^{(n)}}{h} \]
	giving
	\[ \hat{u}_{k}^{(n+1)} = (1 - \sigma ) \hat{u}_{k}^{(n)} + \sigma e^{-2\pi i k/N} \hat{u}_{k}^{(n)}  
	= (1 - \sigma + \sigma e^{-2\pi i k/N})\hat{u}_{k}^{(n)} .
	\]
	
	\pause
	Going through the same calculation gives 
	\[ 0 \leq \sigma \leq 1. \]
	
	\pause
	One would think that the central difference discretization would give better results but a simimlar analysis shows that it's actually \emph{unconditionally unstable}. This is also true for doing the calculation in Fourier space. These really are hard equations.
\end{frame}

\begin{frame}{Upwind scheme}
	Let's generalize the problem a little bit and assume that $ c $ depends on the spatial point. We can discretize the system using an implicit scheme wherever $ c(x) < 0 $ and an explicit scheme when $ c(x) > 0 $. This is called \alert{upwinding}. 
	
	\pause
	This can be generalized to higher dimensions. The name comes from the fact that we define the finite difference stencil in the opposite direction of $ c $ i.e. upwind. There are also higher order upwind schemes with larger finite difference stencils but we will not cover that here. 
	
	\pause
	These methods are important for true advection equations where the time evolution of a field is given by
	\[ \frac{D \phi(t,\vx)}{\diff t} = \partial_t \phi + \vec{v} \cdot \nabla \phi \]
	with some velocity field $ \vec{v} $. 
\end{frame}

%\begin{itemize}[<+->]
%\end{itemize}

\end{document}
