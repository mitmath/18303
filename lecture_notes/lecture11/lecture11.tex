% !TEX TS-program = xelatex
% !TEX encoding = UTF-8 Unicode
\input{../structure.tex}

\title{Finite differences in higher dimensions}
\subtitle{Laplace and Poisson equations}
\date{25/3/2021}
\date{}
%\author{18.303 Linear Partial Differential Equations: Analysis and Numerics}
\institute{18.303 Linear Partial Differential Equations: Analysis and Numerics}
\titlegraphic{\hfill\includegraphics[height=2em]{../MIT-logo.pdf}}

\begin{document}
	
	\maketitle
	
%	\begin{frame}{Table of contents}
%		\setbeamertemplate{section in toc}[sections numbered]
%		\tableofcontents%[hideallsubsections]
%	\end{frame}

\begin{frame}{1d Poisson equation}
	\begin{block}{\centering Poisson's equation with Dirichlet boundaries}
		\begin{align*}
			\dd[2]{u(x)}{x} &= f(x), \\
			u(0) = &u_0, \; u(L) = u_{N+1}.
		\end{align*}
	\end{block}	

Here $ x \in (0,L) $. We discretize the space $ x $ as before: $ x_k = k \Delta x  $ and $ u_k = u(x_k) $, where $ k = 1,2,...,N $ and $ \Delta x = L/(N+1) $. 

We define the Dirichlet Laplacian
\[ D^{(x)} = \frac{1}{\Delta x^2} \begin{pmatrix}
	-2 & 1 &  &   & &  &  \\
	1 & -2 & 1  &   & &  &  \\
	& & \ddots & \ddots & \ddots  & & \\
	& & & & 1 & -2 & 1 \\
	& & & & & 1 & -2 \\
\end{pmatrix} \in \R^{N \times N}. \]
\end{frame}

\begin{frame}
	In the discrete setting we have the equation
	\[ D^{(x)} \vu  = \fone, \]
	but we are missing something? \pause Boundaries.
	
	\pause
	Looking at the first equation we have 
	\[ \frac{1}{\Delta x^2}(u_0 - 2 u_1 + u_2) = f_1 \]
	while the last equation gives
	\[ \frac{1}{\Delta x^2}(u_{N-1} - 2 u_{N} + u_{N+1}) = f_N. \]
	
	\pause
	We can move the known constants to the RHS of the equation. Now we get 
	\[ D^{(x)} \vu  = \fone_\text{B}, \]
	where 
	\[ \fone_\text{B} = \fone - \vec{b}. \]
	
	\pause
	The first entry of $ \vec{b} = u_0/\Delta x^2 $ and the last one is $ u_{N+1}/\Delta x^2 $. Apart from that, it's zero.
\end{frame}

\begin{frame}
	How does one solve for the Laplace equation 
	\[ \dd[2]{u(x)}{x} = 0. \]
	in this setting?
	
	\pause
	We can simply set $ \fone = 0 $ and since $ D^{(x)}  $ is invertible, the equation can be solved.
\end{frame}

\begin{frame}{Two dimensions}
	Now we have 
	\[ \begin{split}
	&\Delta u(x,y) = f(x,y), \\
	&u(x,0) = u^\text{(b)}(x), u(x,L_y) = u^\text{(t)}(x), \\
	&u(0,y) = u^\text{(l)}(y), u(L_x,y) = u^\text{(r)}(y). \\
	\end{split}
	\]
	
	\pause
	We discretize the space: $ x_i = x(i \Delta x) $, $ y_j = y(j\Delta y) $, $ u_{i,j} = u(x_i,y_j) $ and so on. Here $ i=1,2,...,N $ and $ j=1,2,3,...,M $. 
	$ \Delta x = L_x/(N+1) $ and $ \Delta y = L_y/(M+1) $. 
\end{frame}

\begin{frame}
	We can write the Laplace equation in indices as 
	\[ \frac{1}{\Delta x^2}(u_{i-1,j} - 2 u_{i,j} + u_{i+1,j})  
	+ \frac{1}{\Delta y^2}(u_{i,j-1} - 2 u_{i,j} + u_{i,j+1}) = f_{i,j}.
	\]
	Here again $ i = 1,2,...,N $ and $ j = 1,2,3,...,M $. What do we do with boundaries?
	
	\pause
	We solve them in a similar way. The boundary vector is in indices
	\[ \frac{1}{\Delta x^2}(\delta_i^1 u_{0,j} + \delta_i^N u_{N+1,j} ) 
	+ \frac{1}{\Delta y^2}(\delta_j^1 u_{i,0} + \delta_j^M u_{i,N+1} )
	= \frac{1}{\Delta x^2}(\delta_i^1 u_{j}^{(\text{l})} + \delta_i^N u_{j}^{(\text{r})} ) 
	+ \frac{1}{\Delta y^2}(\delta_j^1 u_{i}^{(\text{b})} + \delta_j^M u_{i}^{(\text{t})} )
	\]
	
	\pause
	This will give a boundary matrix $ b_{i,j} $ that has the discretized boundary function values on the boundary. Now the source for the equation $ f^\text{(B)} = f - b $.
\end{frame}

\begin{frame}
	We can regroup the LHS of the equation
	\[ \frac{1}{\Delta x^2}(u_{i-1,j} - 2 u_{i,j} + u_{i+1,j}) -  \frac{2 u_{i,j}}{\Delta y^2}
	+ \frac{1}{\Delta y^2}(u_{i,j-1} + u_{i,j+1}) = f^\text{(B)}_{i,j}.
	\]
	We notice that we have four terms for the column $ j $ on the left and two other terms for columns $ j-1 $ and $ j+1 $. Which linear operator gives the term in the first parentheses?
	
	\pause
	It is exactly the matrix $ D^{(x)} $. 
	
	\pause
	Let us denote the $ j $th column of $ u $ as $ \vu_j $ and the same for $ f^\text{(B)} $. Now we have
	\[ D^{(x)} \vu_j -  \frac{2}{\Delta y^2} \vu_j
	+ \frac{1}{\Delta y^2}(\vu_{j-1} + \vu_{j+1}) = \fone^\text{(B)}_j. \]
	
	\pause
	Let us define $ I_y = I/\Delta y^2 $. Now we have
	\[ I_y \vu_{j-1} +  \underbrace{\left(D^{(x)} - 2 I_y \right)}_{=B} \vu_j + I_y \vu_{j+1} = \fone^\text{(B)}_j. \]
\end{frame}

\begin{frame}
	\[ I_y \vu_{j-1} +  B \vu_j + I_y \vu_{j+1} = \fone^\text{(B)}_j. \]
	Doesn't this look familiar?
	
	\pause
	We can make a matrix of matrices $ A $ called a block matrix. The equation will look like this
	\[ A U = F^{(\text{B})}, \]
	where
	\[ A = \begin{pmatrix}
		B   & I_y & && \\
		I_y & B   & I_y && \\
		    & \ddots & \ddots & \ddots& \\
		    &    & I_y & B   & I_y \\
		    &      &    & I_y & B   \\
	\end{pmatrix} \]

	\pause
	Note that this has the Dirichlet conditions in it. How many $ B $s are there in $ A $? \pause The dimension of $ y $ discretization i.e. $ M $. $ A $ is called a block tridiagonal matrix.
\end{frame}

\begin{frame}
	\[ A U = F^{(\text{B})}. \]
	
	How should we interpret $ U $ and $ F^{(\text{B})} $? If we write $ A $ by just filling in the matrices inside it, it will give $ A $ the dimensions $ NM \times NM $. We can then express $ U $ as 
	\[ U = \begin{pmatrix}
		u_{1,1} \\
		u_{2,1} \\
		\vdots \\
		u_{N,1} \\
		u_{1,2} \\
		\vdots \\
		u_{N,M}
	\end{pmatrix}. \]
	
	\pause
	In indices we have 
	\[ U_{N(j-1) + i} = u_{i,j}. \] 
	
	\pause
	The vector $ F^{(\text{B})} $ is flattened in the same way.
\end{frame}

\begin{frame}{Remarks}
	\begin{itemize}[<+->]
		\item Notice that for general Dirichlet (or Neumann) boundary conditions we can't use separation of variables. 
		\item This makes it hard to solve these problems analytically. 
		\item Moreover, it can be hard to express $ u $ in any known eigenbasis making it hard to use \alert{spectral methods}.
		\item Finite difference methods are easy to implement for simple geometries.
		\item The discrete linear equations can be solved efficiently using existing packages. 
		\item Can be extended to non-uniform meshes but that requires quite a bit of effort.
	\end{itemize}
\end{frame}

\end{document}
