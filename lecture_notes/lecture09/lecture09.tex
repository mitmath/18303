% !TEX TS-program = xelatex
% !TEX encoding = UTF-8 Unicode
\input{../structure.tex}

\title{Wave equation and resonances}
\subtitle{How radio works etc...}
\date{25/3/2021}
\date{}
%\author{18.303 Linear Partial Differential Equations: Analysis and Numerics}
\institute{18.303 Linear Partial Differential Equations: Analysis and Numerics}
\titlegraphic{\hfill\includegraphics[height=2em]{../MIT-logo.pdf}}

\begin{document}
	
	\maketitle
	
%	\begin{frame}{Table of contents}
%		\setbeamertemplate{section in toc}[sections numbered]
%		\tableofcontents%[hideallsubsections]
%	\end{frame}

\begin{frame}{Laplace transform}
	Last time we talked about the Fourier transform when we were dealing with a function (time evolution of the Fourier coefficients for the wave equation) that is defined on the real line $ \R $. 
	
	\pause
	Often with problems like this we are actually interested in how a system responds to some sort of input starting at time $ t=0 $. This gives the domain $ \R_{+} $ i.e. the positive real numbers (normally including $ t=0 $). 
	
	\pause
	The natural tool for studying these sort of systems e.g. the time evolution of the wave equation is the \alert{Laplace transform}. 
\end{frame}

\begin{frame}{Laplace transform}
	We define a linear transformation $ \opone:  V \to W $, where $ V $ and $ W $ are suitable spaces of functions. We write
	\[ \opone[f](s) = \int_{0}^{\infty} e^{-st} f(t) \diff t =: F(s). \]
	
	\pause
	This transform has some nice properties. For example, for derivatives of a function we have
	\[ \opone [f'](s) = \int_{0}^{\infty} e^{-st} f'(t) \diff t \]
	
	\pause
	Integrating by part gives 
	\[ e^{-st} f(t)\biggr|_{t=0}^{\infty} + s \int_{0}^{\infty} e^{-st} f(t) \diff t = s F(s) - f(0^{-}). \]
	Here $ f(0^{-}) = \lim_{t\to 0^{-}} f(0) $ is the left limit at zero. For continuous functions it is just $ f(0) $. 
\end{frame}

\begin{frame}
	Can you generalize this result for higher order derivatives? Try it for the second order.
	
	\pause
	\[ \begin{split}
	\opone [f^{(n)}](s) &= \int_{0}^{\infty} e^{-st} f^{(n)}(t) \diff t   \\
	& = e^{-st} f^{(n-1)}(t)\biggr|_{t=0}^{\infty} + s \int_{0}^{\infty} e^{-st} f(t)^{(n-1)} \diff t \\
	&=   s \opone [f^{(n-1)}](s)-f^{(n-1)}(0).
	\end{split}
	\]
	
	\pause
	Using the above formula repeatedly gives
	\[ \opone [f^{(n)}](s) = s^{n} F(s) - s^{n-1} f(0) - s^{n-2} f'(0) - ... - f^{(n-1)}(0).  \]
	
	\pause
	We see immediately that this is perfect for initial value problems where derivatives at $ 0 $ are given. 
\end{frame}

\begin{frame}{Connection with Fourier transform}
	We see that 
	\[ F(2\pi i \xi) = \int_{0}^{\infty} e^{-2\pi i \xi t} f(t) \diff t = \hat{f}(\xi) \]
	for functions with $ f(t) = 0  $ if $ t<0 $.
	
	\pause
	\textbf{Strengths of Laplace transform:}
	\begin{itemize}[<+->]
		\item The transform is real. This is sometimes very handy.
		\item The transform exists for a larger class of functions since the Laplace integral converges very strongly.
		\item Can be evaluated with efficient algorithms for Fourier transform.
	\end{itemize}
\end{frame}

\begin{frame}
	For example, let us calculate the Laplace transform for a function that is $ 0 $ for $ t<0 $ and $ \sin(\omega t) $ for $ t>0 $:
	\[ \opone[\sin(\omega t)](s) = \int_{0}^{\infty} e^{-st} \sin(\omega t) \diff t \]
	
	\pause
	Casting the $ \sin $ function in the exponential form gives 
	\[  
	\begin{split}
		\int_{0}^{\infty} e^{-st} \frac{1}{2i} \left( e^{i \omega t} - e^{-i\omega t} \right) \diff t
		 &= \frac{1}{2i} \int_{0}^{\infty} e^{(i \omega - s)t } - e^{-(i\omega + s)t } \diff t \\
		 &= \frac{1}{2i} \left( \frac{1}{s-i\omega} - \frac{1}{s + i \omega} \right) \\
		 &= \frac{\omega}{s^2 + \omega^2}.
	\end{split}
	\]
\end{frame}

\begin{frame}
	We can try doing the same for the Fourier transform:
	\[
	\begin{split}
	\fouriert[\sin(\omega t)](\xi) &= \int_{0}^{\infty} e^{-2\pi i \xi t} \sin(\omega t) \diff t \\
	&= \frac{1}{2i} \int_{0}^{\infty} e^{(\omega - 2\pi \xi )i t} - e^{-(\omega + 2\pi \xi )i t} \diff t.
	\end{split}
	\]
	
	\pause
	Since the integrand is oscillating infinitely, this integral doesn't converge in the traditional sense. 
	
	\pause
	However, this integral can be regularized by considering $ \fouriert[\sin(\omega t)](\xi - i\epsilon) $ and taking $ \epsilon \to 0 $.
	
	\pause
	This gives 
	\[ \fouriert[\sin(\omega t)](\xi) = \frac{1}{2i} \left( \frac{1}{2\pi i \xi - i \omega}
	- \frac{1}{2\pi i \xi + i \omega  }\right) = \frac{\omega}{\omega^2 - (2\pi \xi)^2}.
	 \]
	
	\pause
	We can recognize this as the Laplace transform evaluated at $ s = 2\pi i \xi $ just as it should be.
\end{frame}

\begin{frame}
	This brings us to the \textbf{downsides of Laplace transform:}
	\begin{itemize}[<+->]
		\item Higher dimensional generalizations are not so useful since we hardly need the positive quadrant of $ \R^n $.
		\item Inverse transform requires extending the Laplace transform $ F(s) $ to the complex plane and sometimes requires some regularization magic.
	\end{itemize}
\end{frame}

\begin{frame}{Inverse Laplace transform}
	Let's use the relationship between Fourier and Laplace transforms 
	\[ F(2\pi i \xi) = \int_{0}^{\infty} e^{-2\pi i \xi t} f(t) \diff t = \hat{f}(\xi) \]
	to derive the inverse transform.
	
	\pause
	Taking the inverse Fourier transform gives 
	\[ f(t) = \int_{-\infty}^{\infty} e^{2\pi i \xi t} F(2\pi i \xi) \diff \xi. \]
	
	\pause
	Making the change of variables $ s = 2\pi i \xi $ gives
	\[ 
	f(t) = \frac{1}{2\pi i} \int_{-i\infty}^{i\infty} e^{s t} F(s) \diff s.
	\]
	
	\pause
	We see that the integral has to be evaluated on the imaginary axis of the complex plane. 
\end{frame}

\begin{frame}
	Let us try to calculate the inverse Laplace transform of the function
	\[ F(s) = \frac{\omega_0}{s^2 + \omega_0^2}. \]
	
	\pause
	We know from before that this is the Laplace transform of $ \sin (\omega_0 t) $. We have
	\[ \frac{1}{2\pi i} \int_{-i\infty}^{i\infty} e^{st}\frac{\omega_0}{s^2 + \omega_0^2} \diff s.   \]
	
	\pause
	Making the change of variable $ s = i \omega $ gives
	\[ \frac{1}{2\pi} \int_{-\infty}^{\infty} e^{i \omega t} \frac{\omega_0}{\omega_0^2 - \omega^2} \diff \omega = -\frac{1}{2\pi} \int_{-\infty}^{\infty} e^{i \omega t} \frac{\omega_0}{(\omega - \omega_0)(\omega + \omega_0)} \diff \omega. \]
	
	\pause
	We notice that the integral has two singularities at $ \omega = \pm \omega_0 $ and thus can't be in general calculated.
\end{frame}

\begin{frame}
	The answer comes from regularization. If we had instead
	\[ -\frac{1}{2\pi} \int_{-\infty-i\epsilon}^{\infty-i\epsilon} e^{i \omega t} \frac{\omega_0}{(\omega - \omega_0)(\omega + \omega_0)} \diff \omega \]
	we wouldn't be integrating over the singularities on complex plane.
	
	\pause
	The above expression can be turned into a contour integral on complex plane so in fact it doesn't matter how large $ \epsilon $ is. 
	
	\pause
	This integral is evaluated by the residues i.e.
	\[ f(t) =  -\frac{1}{2\pi} \oint_\gamma g(w) \diff \omega 
	= \frac{1}{i} \left[ \resid(g,\omega_0) + \resid(g,-\omega_0)  \right]  \]
	
	\pause
	Evaluating this gives
	\[ f(t) = \frac{\omega_0}{i} \left(\frac{e^{i\omega_0 t}}{2\omega_0} - \frac{e^{-i\omega t}}{2\omega_0}   \right)  = \sin(\omega_0 t).\]
	
	If $ t<0 $ it turns out that this integral gives 0, as it should.
\end{frame}

\begin{frame}{Inverse Laplace transform}
	The formula we obtained can be modified to work for these difficult functions. In general we have 
	\[ \opone^{-1} [F](t) = \frac{1}{2\pi i} \int_{-i\infty + \Delta}^{i\infty + \Delta} e^{st} F(s) \diff s, \]
	where $ \Delta $ is a real number s.t. it is larger than any real parts of singularities of $ F(s)  $ on the complex plane. 
	
	\pause
	So, in short, inverse Laplace transforms are a bit more complicated. However, if $ F $ is an entire function on $ \CC $ without singularities, we can set $ \Delta =0 $. 
\end{frame}

\begin{frame}{Resonances}
	Consider the ODE 
	\[ u''(t) + \lambda^2 u(t) = \sin(\omega_0 t), \]
	with initial conditions $ u'(0)=u(0) = 0 $. With different initial conditions we would have to be concerned with the solution to the homogeneous equation, which we know very well how to do at this point.
	
	\pause
	Notice here that if $ \lambda \to \eigvone $, this is the equation we obtain for each eigenvalue of the inhomogeneous wave equation (possibly there's something time independent multiplying the RHS). 
	
	\pause
	In general this is solved by an ansatz \[ u(t) = A \sin(\omega_0 t) + B \cos(\omega_0 t). \]
	Plugging this in gives 
	\[  (\lambda^2 -\omega_0^2) (A \sin(\omega_0 t) + B \cos(\omega_0 t)) = \sin(\omega_0 t). \]
\end{frame}

\begin{frame}
	From this we can solve 
	\[ A = \frac{1}{\lambda^2-\omega_0^2}, \; B=0, \]
	as long as $ \omega_0 \neq \pm \lambda $.
	
	\pause
	What if $ \omega_0 = \lambda $?
	
	\pause
	In this case this is solved by an ansatz $ u(t) = A t \sin(\omega_0 t) + B t \cos(\omega_0 t)  = t(A  \sin(\lambda t) + B  \cos(\lambda t)) $. Plugging this in gives
	\[ -t \lambda^2 (A  \sin(\lambda t) + B  \cos(\lambda t) ) 
	+ 2 \lambda (A \cos(\lambda t) - B \sin(\lambda t)  ) + t\lambda^2 (A  \sin(\lambda t) + B  \cos(\lambda t)) = \sin(\lambda t). \]
	
	\pause
	Equating this with the RHS gives 
	
	\pause
	\[ A= 0, B = -\frac{1}{2 \lambda} \]
	and the solution is 
	\[ u(t) = -\frac{t}{2 \lambda} \cos(\lambda t). \]
\end{frame}

\begin{frame}
	Notice how the amplitude of this solution grows linearly in time. This is called a resonant frequency. 
	
	\pause
	Imagine that we would have a wave equation (we'll do this in detail later) with eigenvalues 
	$ \eigvone $. If we have an input that is oscillating at a frequency $ f = \eigvone/(2\pi) $ the system starts to vibrate uncontrollably. 
	
	\pause
	Radio is an example of a system where this phenomenon is made useful. Radio circuit is basically an electronic oscillator whose eigenmodes can be modified. By tuning the eigenmode to match the external frequency the oscillations increase greatly and we can listen to a radio channel at a given frequency. 
	
	\pause
	Of course nothing blows up in nature. Next time we will talk about softening of resonances due to some sort of damping in the system. We will also derive the results above using the Laplace transform. It is recommended to revise complex analysis for this. 
\end{frame}

\end{document}
