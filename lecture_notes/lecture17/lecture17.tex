% !TEX TS-program = xelatex
% !TEX encoding = UTF-8 Unicode
\input{../structure.tex}

\title{Green's functions}
\subtitle{Direct solution to PDEs}
\date{25/3/2021}
\date{}
%\author{18.303 Linear Partial Differential Equations: Analysis and Numerics}
\institute{18.303 Linear Partial Differential Equations: Analysis and Numerics}
\titlegraphic{\hfill\includegraphics[height=2em]{../MIT-logo.pdf}}

\begin{document}
	
	\maketitle
	
%	\begin{frame}{Table of contents}
%		\setbeamertemplate{section in toc}[sections numbered]
%		\tableofcontents%[hideallsubsections]
%	\end{frame}

\begin{frame}{Green's function methods}
	Consider a PDE
	\[ \opone u(\vx) = f(\vx), \]
	where $ \vx \in \Omega \subset \R^N $ and $ \opone $ is some linear differential operator. Furthermore, let's assume boundary conditions
	\[ \opthr u(\vx) = g(\vx) \]
	on the boundary $ \partial \Omega $.
	
	\pause
	The underlying idea of Green's function methods is to find the the solution by inverting $ \opone $ as
	\[ u(\vx) = \optwo f(\vx) := \opone^{-1} f(\vx). \]
	
	\pause
	Some remarks:
	\begin{itemize}[<+->]
		\item $ \optwo f(\vx) $ is a linear operation but in general $ \optwo $ is not a function
		\item The solution is expressed as $ u(\vx) = \optwo f(\vx) = \int_{\Omega} G(\vx,\vy) f(\vy) \diff \vy $, where $ G $ is the \alert{Green's function}
		\item $ G $ will also depend on the boundary condition
	\end{itemize}
\end{frame}

\begin{frame}{Properties of the Green's function}
	The operation $ \optwo f $ is a linear operation. We also have
	\[ \optwo [\opone u](\vx) = \int_{\Omega} G(\vx,\vy) [\opone u](\vy) \diff \vy = 
	\int_{\Omega} G(\vx,\vy) f(\vy) \diff \vy
	= u(\vx) \]
	it follows that $ \opone G = \delta(\vx - \vy) $. 
	%following from the fact that $ \optwo = \opone^{-1} $.
	
	\pause
	This can really be thought of as a generalization of solving a linear system: now the linear operations are generalizations of matrix-vector products, we just replace the discrete indices with a continuous $ \vx $ and $ \vy $. 
\end{frame}

\begin{frame}{Example}
	Let's calculate the Green's function for a 1d Laplacian on the real line $ \R $. We have
	\[ \partial_x^2 G(x,y) = \delta(x-y). \]
	
	\pause
	The Laplacian is translationally symmetric i.e. $ \partial_x^2 u(x) \Big|_{x=x_0} = \partial_x^2 u(x+c) \Big|_{x=x_0-c}  $ for all $ c \in \R$. This means that the operator doesn't explicitly depend on the point it's evaluated at. This implies that also the Green's function has to be translationally invariant. Thus we can write $ G(x,y) = G(x-y) $.
	
	\pause
	We can calculate the Fourier transform of the equation for the Green's function giving 
	\[ \int_{\R}  e^{-i k (x-y)} \partial_x^2 G(x-y) \diff x = \int_{\R} e^{-i k (x-y)} \delta(x-y) \diff x = 1. \]
	
	\pause
	We know that the Fourier transform of $ f'(x) $ is $ ik\hat{f}(k) $ giving the LHS. We have $ -k^2 \hat{G}(k) = 1 $.
\end{frame}

\begin{frame}
	\begin{columns}[T,onlytextwidth]
		\column{0.5\textwidth}
		Now we get 
		\[ G(x-y) = -\frac{1}{2\pi} \int_{\R} \frac{e^{ik (x-y)}}{k^2} \diff k. \]
		
		We can evaluate this integral as a limit of the contour shown on the right. We have 
		\[ G(x-y) = \int_{\tilde{\R}}g(k)... + \int_{C_R}... +\int_{C_{\epsilon}}... - \int_{C_{\epsilon}}... \]
		since the integral on the path $ C_R $ goes to zero as $ R \to \infty $. The semicircle $ C_{\epsilon} $ gives half of the Cauchy integral while the rest give the complete integral. 
		
		In the end we have 
		\[ G(x,y) = \pi i \resid(g,0). \]
		
		\column{0.5\textwidth}
		\vspace{4em}
		\begin{figure}
			\centering
			\includegraphics[width=0.8\linewidth]{contour.pdf}
			\caption{The contour for integration.}
		\end{figure}
	\end{columns}
\end{frame}

\begin{frame}
	We get 
	\[ G(x-y) = -\frac{i}{2} \dd{e^{ik(x-y)}}{k}\Bigg|_{k=0} = \frac{1}{2}\begin{cases}
		x-y, & x>y\\
		y-x, & x<y
	\end{cases} = \frac{1}{2} |x-y|. \]
	
	\pause
	It is expected that we have something symmetric since $ \optwo $ has to be self-adjoint (like the Laplacian). 
	
	\pause
	We can check the solution by calculating
	\[ u(x) = \int_{\R} G(x-y) f(y) \diff y = \frac{1}{2} \int_{\R} |x-y| f(y) \diff y. \]
	
	\pause
	Integrating by parts gives 
	\[ u(x) = \frac{1}{2} \int_{\R} \operatorname{sgn}(x-y) \int_{-\infty}^{y}f(y') \diff y' \diff y. \]
	
	\pause
	Here we assume that $ f(y) $ is integrable implying that the boundary term vanishes. 
\end{frame}

\begin{frame}
	Calculating integration by parts again gives
	\[ u(x) = \frac{1}{2} \int_{\R} 2 \delta(x-y) \int_{-\infty}^{y} \int_{-\infty}^{y'} f(y'') \diff y'' \diff y' \diff y. \]
	
	\pause
	Here we notice that in distribution sense (weak sense), the derivative of the $ \operatorname{sgn}(x-y) $ function is $ -2 \delta(x-y) $. 
	
	\pause
	We can get rid of one integral using the delta distribution giving
	\[ u(x) =  \int_{-\infty}^{x} \int_{-\infty}^{y'} f(y'') \diff y'' \diff y'. \]
	
	\pause
	This is just the double integral of $ f $, which makes sense since we wanted to solve
	\[ \partial_x^2 u(x) = f(x). \]
\end{frame}

\begin{frame}
	For 2d Laplacian we have 
	\[ G(r) = \frac{1}{2\pi} \log(r) \]
	and in 3d
	\[ G(r) = -\frac{1}{4\pi r}. \]
	Here $ r = |\vx - \vy| $.
	
	\pause
	\textbf{Important:} these equations are for the whole space: boundary conditions will change the Green's function. 
\end{frame}

\begin{frame}{Example with boundaries}
	Assume we have a Dirichlet problem for the Laplacian i.e.
	\[ \Delta u(\vx) = f(\vx), \]
	when $ \vx \in \Omega \subset \R^n $ with boundary condition
	\[ u(\vx) = g(\vx) \]
	on the boundary $ \vx \in \partial \Omega $.
	
	\pause
	Assume we know a function for which 
	\[ \Delta_\vx G(\vx-\vy) = \delta(\vx-\vy). \]
\end{frame}

\begin{frame}
	We have 
	\[ u(\vx) = \int_{\Omega} \delta(\vx-\vy) u(\vy) \diff \vy =  \int_{\Omega} \Delta_\vx G(\vx-\vy) u(\vy) \diff \vy. \]
	
	\pause
	We can change the derivative to act on $ \vy $ since $ G $ is a function of the distance $ \vx - \vy $. Integrating by parts gives
	\[ u(\vx) = \int_{\partial \Omega} \underbrace{u(\vy)}_{=g(\vy)} \nabla_{\vy} G \cdot \diff \vec{S} -  
	\int_{\Omega} \nabla_{\vy} G \cdot \nabla_{\vy}  u(\vy) \diff \vy.
	\]
	
	\pause
	After integrating by parts again we get 
	\[ u(\vx) = \int_{\partial \Omega} g(\vy) \nabla_{\vy} G \cdot \diff \vec{S}  
	- \int_{\partial \Omega}   G \nabla_{\vy} u(\vy) \cdot \diff \vec{S}
	+ \int_{\Omega}  G   \underbrace{\Delta_{\vy} u(\vy)}_{=f(\vy)} \diff \vy.
	\]
	
	\pause
	The second term is zero since $ G(\vx-\vy) = 0 $ if either $ \vx $ or $ \vy $ is on the boundary (we don't solve for the boundary points). 
\end{frame}

\begin{frame}
	We get 
	\[ u(\vx) = \int_{\partial \Omega} g(\vy) \nabla_{\vy} G \cdot \diff \vec{S}  
	+  \int_{\Omega}  G(\vx -\vy)   f(\vy) \diff \vy.
	\]
	
	Notice how the boundary plays a role here. 
\end{frame}

\begin{frame}{Eigenbases and the Green's function}
	We have a Poisson type equation 
	\[ \opone u(\vx) = f(\vx) \]
	on some bounded domain $ \vx \in \Omega $. 
	
	\pause
	Assume we have a enumerable orthonormal eigenbasis for the operator $ \opone $ i.e. 
	\[ \opone \eigtwo(\vx) = \eigvone \eigtwo(\vx) \]
	for some multi index $ \vn $.
	
	\pause
	We have 
	$ \coefone = \coeftwo/\eigvone $. Calculating the solution $ u $ gives
	\[ u(\vx) = \sum_{\vn} \frac{\coeftwo}{\eigvone} \eigtwo (\vx). \]
	
	\pause
	On the other hand we have 
	\[ \coeftwo = \iprod{\eigtwo}{f} = \int_{\Omega} \eigtwo(\vy) f(\vy) \diff \vy. \]

\end{frame}

\begin{frame}
	Plugging this in gives
	\[ u(\vx) = \sum_{\vn} \frac{1}{\eigvone} \int_{\Omega} \eigtwo(\vy) f(\vy) \diff \vy \eigtwo (\vx). \]
	
	\pause
	Changing the order between summation and integration results in 
	\[ u(\vx) = \int_{\Omega} \left( \sum_{\vn} \frac{\eigtwo(\vy) \eigtwo (\vx) }{\eigvone} \right)  f(\vy) \diff \vy,  \]
	where we identify the Green's function in the brackets. 
	
	\pause
	This shows that in principle we can calculate the Green's function if we have solved the eigenproblem. 
\end{frame}

\end{document}
